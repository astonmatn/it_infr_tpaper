\section{Einleitung}
Jeder nutzt es, sei es nur um Inhalte zu konsumieren, wie die Tagesschau oder die Lindenstraße. Aber auch verfolgen wir Sportgroßereignisse wie die Fußball Weltmeisterschaft oder die Olympischen Spiele konsumieren wir auf diesem Weg.
Bereitgestellt werden die Inhalte live und linear, oder auf Abruf. \newline
Der Fernseher bringt die weite Welt und deren Ereignsse ins heimsische Wohnzimmer.
Wegzudenken ist das fernsehen aus der heutigen Gesellschaft nicht mehr, es wird mittlerweile sogar als ein ''Grundbedürfnis'' \footcite[Vgl. ][Sozialgericht Frankfurt, Urteil vom 28. Mai 2009 - S 17 AS 388/06]{sg.TV} eingestuft.
Es zählt zu einem der wichtigsten konsumierten Medien.
Wie in jedem Bereich des täglichen Lebens erhält die Digitalisierung als auch der Fortschritt mal mehr oder weniger schnell Einzug. Aber Sie kommt. In der vorliegenden Hausarbeit wird das Thema
''Streaming und Pay-TV'' betrachtet. \newline
Im speziellen behandelt diese Ausarbeitung die Fragen nach den technischen Grundlagen für das Fernsehen über das Internet als auch für das Pay-TV.
Zusätzlich wird erläutert über welche technischen Erweiterungen der ''Conditional Access'' verfügbar ist.
Ebenso wird erörtert, welche Auswirkungen und Konsequenzen sich für Zuschauer und Programmgestalter, seit und durch die Einführung der neuen oder anderen Möglichkeiten, ergeben oder bereits ergeben haben.

\subsection{Zielsetzung}
Ziel dieser Arbeit ist es Grundlagenwissen zu vermitteln. Ich zeige die grundsätzlichen Unterschiede zwischen dem Konsum von klassischen ''linearen''-Fernsehen, dem Pay-TV und von Inhalten per Streaming auf.
Ich werde eine kleine Übersicht über die Diensteanbieter liefern, ohne auf die Vor- als auch Nachteile der einzelnen Angebote einzugehen, sondern den Focus eher auf die technische Seite legen.


\subsection{Aufbau der Arbeit}
Das Kapitel 2 beinhaltet die technischen Grundlagen der jeweiligen ''Bezugsquelle''. Im Einzelnen gehe ich auf, erforderlich für das Streaming, die technische Übermittlung der Daten über das Internet ein.
Ebenso erläutere ich die Unterschiede im DVB (Digital Video Broadcast). Dort unterscheidet man drei Bezugsquellen und einhergehend Techniken.
In Kapitel 3 werden die Begriffe Pay-TV (siehe Punkt 2.1), Streaming (siehe Punkt 2.2) und lineares TV definiert und Unterschiede aufgezeigt.
Anschließend folgt eine Zusammenfassung der Erkenntnisse, um dann abschließend mit dem Fazit einen kleinen Ausblick in die Zukunft zu geben.







%---------------------------------------------------------
% Einbetten einer Grafik
%---------------------------------------------------------

 \begin{figure}[H]
 \begin{center}
 \includegraphics[width=0.9\textwidth]{fomLogo}
 \caption{Logo der FOM}
 \end{center}
 \end{figure}
%---------------------------------------------------------
% Einbetten einer Fusszeile mit Vgl. zu einem Buch
%---------------------------------------------------------
 So bindet man Fusszeilen ein die gleichzeitig die Bücher aus der literatur.bib in das Verzeichnis schreiben! \footcite[Vgl. ][Seite 1]{Tanenbaum.2003} .
%---------------------------------------------------------
% Einbetten eines Links
%--------------------------------------------------------- 
 so entstehen links \href{https://github.com/astonmatn/it_infr_tpaper.git}{github}
%---------------------------------------------------------
% Einbetten einer Liste
%---------------------------------------------------------
So erstellt man eine Dot-Liste:
\begin{itemize}
\item Butter
\item Mehl
\end{itemize}

%---------------------------------------------------------
% Einbetten einer Aufzählung
%---------------------------------------------------------

So erstellt man eine Aufzählung
\begin{enumerate}
\item erstens
\item zweitens
\item drittens
\end{enumerate}

%---------------------------------------------------------
% Einbetten einer URL
%---------------------------------------------------------
\url{https://www.apple.com}

\enquote{Dies ist eine Quote, wichtig ist hier die Anschliessende Fusszeile}\footcite[Vgl. ][Seite 111]{stern.0117}
