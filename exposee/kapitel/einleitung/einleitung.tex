\section{Einleitung}
Das Thema der vorliegenden Hausarbeit ist \grqq Streaming und Pay-TV\grqq.
Im speziellen behandelt diese Hausarbeit die Fragen nach den technischen Grundlagen für das Fernsehen über das Internet und über welche technischen Erweiterungen der \grqq Conditional Access\grqq  verfügbar ist.
Ebenso erörtert diese Hausarbeit, welche Auswirkunden und Konsequenzen sich für Zuschauer und Programmgestalter ergeben.

\subsection{Zielsetzung}
Ziel dieser Arbeit ist es, Grundlagenwissen zu vermitteln. Ich erörtere die grundsätzlichen Unterschiede zwischen dem klassischen \grqq linearen\grqq--Fernsehen, dem Pay-TV und dem Konsum von TV-Inhalten per Streaming.
Ich werde eine kleine Übersicht über die Diensteanbieter liefern ohne auf die Vor- als auch Nachteile der einzelnen Angebote einzugehen.


\subsection{Aufbau der Arbeit}
Das Kapitel 2 beeinhaltet die technischen Grundlagen der jeweiligen \grqq Bezugsquelle \grqq. Im einzelnen gehe ich auf, erforderlich für das Streaming, die technische Übermittlung der Daten über das Internet ein. Ebenso erläutere ich die Unterschiede im DVB (Digital Video Broadcast), dort unterscheidet man drei Bezugsquellen und Techniken.
Das Kapitel 3 enthält die Definitonen der Begriffe Pay--TV (siehe Punkt 2.1), Streaming (siehe Punkt 2.2) und \grqq linearem\grqq--TV.




\begin{figure}[H]
\begin{center}
\includegraphics[width=0.9\textwidth]{verzeichnisStruktur}
\caption{Verzeichnisstruktur der \LaTeX{}-Datein}
\end{center}
\end{figure}
