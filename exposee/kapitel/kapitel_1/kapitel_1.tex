\newpage
\section{Technische Grundlagen}

Siehe auch Wissenschaftliches Arbeiten~\footcite[Vgl. ][Seite 1]{Balzert.2008}. Damit sollten alle wichtigen Informationen abgedeckt sein ;-)

\subsection{DVB - Digital Video Broadcast}
\subsubsection{DVB-T}
\begin{itemize}
\item DVB-T alt:
\item DVB-T neu:
\end{itemize}
\subsubsection{DVB-S}
\subsubsection{DVB-C}

\subsection{Pay-TV}

\subsection{Streaming}
Trichtermethode: Man beginnt mit der eigentlichen  Konklusion und überlegt dann, welche allgemeinen Teile dafür benötigt werden.

Welchen Mehrwert soll die Arbeit bieten \footnote{Diese Fußnote hat inhaltlich keinen Sinn. Es soll nur ein langer Text generiert werden, dass dieser Vermerk über zwei Zeilen reicht und bündig dargestellt wird.}? Auch darüber nachdenken, wie die Arbeit einen selbst weiter bringen kann. Studienverlauf prüfen. Welche Vorlesungen hat mich besonders interessiert? Wo liegen meine Stärken etc.
